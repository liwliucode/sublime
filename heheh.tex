% !TEX program=xelatex
\documentclass[a4paper,UTF8]{article}
\usepackage{ctex}
\title{sublime text3的一些快捷键学习}
\author{LLW}
\date{}
\begin{document}
如何将文件中的某个单词更改为另一个?

方法一:利用查找替换功能:ctrl + h

方法二(推荐):多行游标功能,选中一个后,按ctrl+d可以同时选中另一个,同时多了另一个光标。

但多行游标能完成查找替换功能不能完成的工作。
hehedahehed heheda 
比如在某些符合条件的语句后面添加新行,同时加入一些新的文本,如何快速的达到这一目的?
- 可以选中某一个模式,然后ctrl+D选中另一个,如果有某些不想添加新行的模式则按ctrl+K,ctrl+D跳过这个进入下一个符合条件的模式行。
- 还可以按Alt + F3快捷键全选所有符合条件的单词,产生多个光标,而不用一个个ctrl+D选中。
- 如果要在每行都加入光标,可以先ctrl+A然后ctrl+shift+L即可。
- 如果想在某个字符的多行后面加上光标,可以将光标放在这个字符后面,按住shift键,然后右键可以向下拖动产生多个光标。

4. 命令模式(应尽可能使用,而不用浪费脑细胞记忆大量命令的快捷键)

比如用ctrl+N新建一个文件后,默认是plain text,没有语法高亮功能,如何设置语法模式?
- 可以通过右下角的语法选择区选择希望设置的语法模式。
- 还有另一种更好的办法,即使用ctrl + shift + P打开命令模式,然后输入set syntax [language]设置为某种语言的语法模式,比如set syntax java则设置为java语法高亮。
- st3支持模糊匹配,你也可以直接输入syntax java或ssjava。
- 若当前已经是某种语言的语法模式,则可以直接输入其它语言进行切换(而不用输入set syntax或syntax了),比如当然为java语法模式,那么直接输入js就可以马上切换为javascript语法模式。

还可以输入minimap隐藏或显示右边的minimap缩影

5. 快速跳转到某一行

按下Ctrl + G,输入行号,可以快速跳转到该行。

6. 快速添加新行

    Ctrl + Enter可以在当前行下新建一行。
    Ctrl + Shift + Enter可以在当前行上面添加一行。

7. 多行缩进

选中多行后按Ctrl + ]可以增加缩进,按Ctrl + [可以减少缩进。
PS:发现用Tab和Shift + Tab也是可以的。

8. 完整拷贝,避免格式错乱

我们发现,在从别的文件中拷贝一段代码过来的时候,多半只是第一行缩进,后面都乱了,这时可以使用Ctrl + Shift + V进行粘贴,可以在粘贴的过程中保持缩进,这时格式都是正确的。

9. 重新打开关闭的标签

在Chrome里面,如果你不小心关闭了某个标签页并想恢复它,你可以按下Shift + Ctrl + T重新打开它。
在ST3中也一样,如果你不小心关闭了某个文件,可以按下Shift + Ctrl + T快速恢复。连续重复该按键,ST将会按照关闭的先后顺序重新打开标签页。

10. 按住shift + ctrl然后按←或→可快速选中一行中的某一部分,连续按扩大选择范围。

比如你需要将某一部分进行注释(ctrl+/)或删除,使用这个功能就很方便。

11. 上下移动行

定位光标或选中某块区域,然后按shift+ctrl+↑↓可以上下移动该行。

12. shift + ctrl + d可快速复制光标所在的一整行,并复制到该行之前。

13. Ctrl+Shift+M:选中花括号里面的全部内容不包括{}。

14. Ctrl+Shift+K:删除整行。

ctrl-shift-p:打开控制面板

ctrl-p:在项目中搜索文件

ctrl-G:跳转到第几行


ctrl-w:关闭当前窗口

ctrl-shift-w:关闭所有窗口

ctrl-shift-v:粘贴并格式化

Ctrl+D:选择单词,重复可增加选择下一个相同的单词

Ctrl+L:选择行,重复可依次增加选择下一行

Ctrl+Shift+L:选择多行(还不会用啊)

Ctrl+Shift+Enter:在当前行前插入新行

Ctrl+X:删除当前行

Ctrl+M:跳转到对应括号

Ctrl+U:软撤销,撤销光标位置

Ctrl+J:选择标签内容

Ctrl+F:查找内容

Ctrl+Shift+F:查找并替换

Ctrl+H:替换

\textbf{\emph{Ctrl+R:前往 method}}

Ctrl+N:新建窗口

Ctrl+K+B:开关侧栏

Ctrl+Shift+M:选中当前括号内容,重复可选着括号本身

Ctrl+F2:设置/删除标记

\textbf{\emph{Ctrl+/:注释当前行}}

Ctrl+Shift+/:当前位置插入注释

Ctrl+Alt+/:块注释,并Focus到首行,写注释说明用的

Ctrl+Shift+A:选择当前标签前后,修改标签用的

F11:全屏

Shift+F11:全屏免打扰模式,只编辑当前文件

Alt+F3:选择所有相同的词

Alt+.:闭合标签

Alt+Shift+数字:分屏显示

Alt+数字:切换打开第N个文件

Shift+右键拖动:光标多不,用来更改或插入列内容

鼠标的前进后退键可切换Tab文件

按Ctrl,依次点击或选取,可需要编辑的多个位置

按Ctrl+Shift+上下键,可替换行

\begin{tabular}{|llllll|}
\hline
函数  & DECRC    &          &          &          &          \\
    & Best     & Median   & Worst    & Mean     & Std      \\
f1  & 0.00E+00 & 0.00E+00 & 0.00E+00 & 0.00E+00 & 0.00E+00 \\
f2  & 5.68E-14 & 5.68E-14 & 2.27E-13 & 9.09E-14 & 1.10E-13 \\
f3  & 3.48E-06 & 3.13E-05 & 3.85E-04 & 5.33E-05 & 8.15E-05 \\
f4  & 1.11E+00 & 1.19E+01 & 9.52E+01 & 2.00E+01 & 2.48E+01 \\
f5  & 8.27E+01 & 9.48E+02 & 2.16E+03 & 9.46E+02 & 6.25E+02 \\
f6  & 0.00E+00 & 0.00E+00 & 5.68E-14 & 2.27E-15 & 2.27E-15 \\
f7  & 2.22E-06 & 7.40E-03 & 3.69E-02 & 1.09E-02 & 1.35E-02 \\
f8  & 2.00E+01 & 2.00E+01 & 2.00E+01 & 2.00E+01 & 2.00E+01 \\
f9  & 1.99E+00 & 3.98E+00 & 1.99E+01 & 5.33E+00 & 3.99E+00 \\
f10 & 4.68E+01 & 7.06E+01 & 1.23E+02 & 7.30E+01 & 1.83E+01 \\
f11 & 7.21E+00 & 1.44E+01 & 1.82E+01 & 1.37E+01 & 3.03E+00 \\
f12 & 1.14E-13 & 1.77E+02 & 4.97E+03 & 8.61E+02 & 1.39E+03 \\
f13 & 1.96E+00 & 3.15E+00 & 4.42E+00 & 3.24E+00 & 6.65E-01 \\
f14 & 1.09E+01 & 1.20E+01 & 1.26E+01 & 1.19E+01 & 4.38E-01 \\
\hline
\end{tabular}

\end{document}
